\documentclass[10pt]{beamer}
\usetheme{Montpellier}
\usepackage[utf8]{inputenc}
\usepackage{amsmath}
\usepackage{amsfonts}
\usepackage{amssymb}
\setbeamertemplate{footline}[frame number]
\author[Ivonne] 
{Ivonne Monter Aldana\\ivonne.monter@cimat.mx}

\institute[VFU] 
{
  CIMAT
}
\title{PACE 2021-Cluster Editing\\Optimización Estocástica}{}
\setbeamercovered{transparent} 
\setbeamertemplate{navigation symbols}{} 
\logo{\includegraphics[height=1cm]{logo.png}}
\date{27 de mayo de 2021} 
\begin{document}
\titlepage
\begin{frame}{Tabla de Contenido}
\tableofcontents
\end{frame}
\AtBeginSection[]
{
  \begin{frame}
    \frametitle{Tabla de Contenido}
    \tableofcontents[currentsection]
  \end{frame}
}
\section{Introducción}
\subsection{Cluster Editing}
\begin{frame}{Cluster Editing}
El problema de Cluster Editing consiste en transformar una gráfica en una unión disjunta de gráficas completas usando un mínimo número de modificaciones de aristas.
\begin{figure}
\includegraphics[scale=0.3]{ej.png}
\end{figure}
\end{frame}
\begin{frame}{Aplicaciones}
\begin{itemize}
\item Detección a gran escala de genes homólogos.
\item Modelar las interacciones entre proteínas.
\item Detección de componentes patógenos.
\item Análisis de redes sociales.
\end{itemize}
\end{frame}
\begin{frame}{Complejidad del problema}
\begin{itemize}
\item En 1986 los matemáticos Mirko Křivánek y Jaroslav Morávek probaron que el problema de Cluster Editing es NP-hard.

\item Para gráficas de a lo más grado dos, es decir caminos y círculos el problema se puede resolver en tiempo polinomial.
\end{itemize}

\end{frame}
\subsection{Complejidad del problema}
\section{Reto}
\begin{frame}{Objetivo}
La tarea es encontrar la mejor solución para cada instancia en un tiempo límite de 10 minutos e imprimir la lista de aristas modificadas en menos de 30 segundos
\begin{figure}
\includegraphics[scale=0.2]{heur.png}
\end{figure}
\end{frame}
\section{Método implementado}
\subsection{Representación de las soluciones}
\begin{frame}{Representación de las soluciones}
\begin{itemize}
\item El número de clusters es a lo más el número de vértices de la gráfica.
\item Dada una gráfica $G(V,E)$ no dirigida, las soluciones al problema se representan como vectores en $\mathbb{N}^{|V|}$, donde cada componente pertenece al conjunto $\{1,2,\ldots,|V|\}$.
\item La dimensión del espacio de búsqueda es $V^{|V|}$.
\begin{figure}
\includegraphics[scale=0.45]{ERS.png}
\end{figure}
\end{itemize}
\end{frame}

\subsection{Data Reduction}
\begin{frame}{Data Reduction}
\begin{itemize}
\item Dos vértices $u$ y $v$ son gemelos si $v\sim u$ y $N(v)=N(u)$.
\item La propiedad de ser gemelos es transitiva.
\item La reducción consiste en compactar conjuntos de vértices gemelos entre si a un solo vértice.
\end{itemize}

\begin{figure}
\includegraphics[scale=0.35]{EVG.png}
\caption{5 y 7 son gemelos y también 1 y 3}
\end{figure}
\end{frame}

\begin{frame}{Gráfica de la reducción de dimensión}
\begin{figure}[H]
\includegraphics[scale=0.16]{GR.png}
\end{figure}
\end{frame}
\subsection{Casos especiales: Gráficas fuertemente conexas}
\begin{frame}{Casos especiales: Gráficas fuertemente conexas}
´\begin{itemize}
\item Algunas instancias corresponden a gráficas en el que el número de aristas es cercano al número de aristas de un gráfica completa.

\item Si le faltan manos del 10\% de aristas para ser una gráfica completa, entonces todos los vértices se colocan en un mismo cluster.
\begin{table}[H]
\resizebox{9cm}{!}{
\begin{tabular}{|r|r|r|r|r|}
\hline
Instancia&$|E|$&$|V|*(|V|-1)/2$&Costo&Costo*\\ \hline
53&70027 & 79800 &9773&8908\\\hline
79&163863 &191890 &28027&27073\\ \hline
131 & 3353846 &4005865& 652019&652019 \\ \hline
133 & 4005420 & 4005865 &445&445  \\ \hline
139 & 3515629 &4238416 &722787&722787\\ \hline
147 &4160168 &4308580 &148412&148412\\ \hline
159 & 3533156 &4317391 &784235& 784235 \\ \hline
161 &4295123 &4317391 &22268& 22268 \\ \hline
\end{tabular}   
}       
\caption{Instancias fuertemente conexas}
\end{table} 
\end{itemize}
\end{frame}
\subsection{Búsqueda local con escala estocástica y evaluación incremental}
\begin{frame}{Búsqueda local}
\begin{itemize}
\item Para cada componente conexa de la gráfica se implementó una búsqueda local con escalada estocástica y evaluación incremental. 
\item La vecindad de una solución consiste en todas las soluciones que difieren en un solo vértice.

\item Para poder hacer la evaluación incremental, se observó que al cambiar un $v$ vértice de cluster $C1$ a un cluster $C2$ se deben sumar y restar ciertas aristas:
\begin{enumerate}
\item - Cada arista que se sumo al agregar $v$ a $C1$ debe ser restada del costo actual.
\item + Cada arista original de $v$ con el resto de vértices de $C1$ debe ser sumada.
\item - Por cada vértice en $C2$ no adyacente a $v$ se debe sumar una arista.
\item + Cada arista original entre v y los vértices de $C2$ debe ser restada.
\end{enumerate}
\end{itemize}
\end{frame}
\begin{frame}{Resultados}

\end{frame}
\end{document}